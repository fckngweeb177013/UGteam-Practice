\documentclass[10pt,pdf,hyperref={unicode}]{beamer}
\usetheme{Frankfurt}
\usepackage{ucs} 
\usepackage[utf8x]{inputenc}
\usepackage[russian]{babel}

\title{Научно-исследовательская практика}
\author{Уткин Артём и Гонтарский Владислав}
\date{1 Июля 2022}
\institute {ИФМНиИТ БФУ им.Канта}
\subtitle{Решение задачи по программированию}

\begin{document}

\begin{frame}
\titlepage
\end{frame}
\section{БФУ им.Канта}

\begin{frame}
\frametitle{Формулировка задачи}
\begin{itemize}
\item Задача
\begin{itemize}
\item Написать программу, вычисляющую n-е по счёту простое число
\end{itemize}
\item Объяснение
\begin{itemize}
\item В первой строке находится ровно одно целое число k, задающее количество чисел в списке. За ним следуют k целых чисел, по одному в строке. Все числа положительные и не превосходят 15000.
\item Для каждого числа n из списка нужно вывести n-е по счёту простое число. Ответ для каждого числа должен находиться в отдельной строке.
\end{itemize}
\end{itemize}
\end{frame}

\begin{frame}
\frametitle{Методы решения}

Одним из возможных методов было создание массива индексов с дальнейшим перебором чисел и проверкой их на простоту с помощью определенной функции. 
\begin{block}{Что же это за функция?}
Суть функции состоит в том, чтобы проверить делимость некоторого числа на все натуральные числа, начиная с двух и заканчивая квадратным корнем самого числа.
\end{block}
Но поскольку требовалось бы повторять операцию для каждого из чисел n, данный вариант был отвергнут в пользу более оптимального.
\end{frame}

\begin{frame}
\frametitle{Выбранный метод}
\begin{figure}
\includegraphics[scale=0.5]{Resheto.jpg}
\end{figure}
\centering
Предпочтительным и оптимальным оказалось решение, использующее Решето Эратосфена.
\end{frame}

\begin{frame}
\frametitle{Описание решения}
Решение состоит в создании массива из переменных типа boolean, в котором все элементы (кроме нулевого и первого) имеют значение true. После, используя Решето Эратосфена, мы будем записывать полученные нами простые числа в другой массив, тем самым получив упорядоченную последовательность простых чисел.
\begin{block}{Как это поможет решить задачу?}
В дальнейшем мы просто можем взять известные нам индексы и получить искомые простые числа.
\end{block}
\end{frame}

\begin{frame}
\frametitle{Код решения}
\begin{figure}
\includegraphics[scale=0.55]{code.jpg}
\end{figure}
\end{frame}

\end{document}